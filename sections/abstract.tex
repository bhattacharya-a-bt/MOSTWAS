\begin{abstract}
Traditional predictive models for transcriptome-wide association
studies (TWAS) consider only single nucleotide polymorphisms (SNPs)
local to genes of interest and perform parameter shrinkage
with a regularization process. These approaches ignore the effect of
distal-SNPs or possible effects underlying the SNP-gene
association. Here, we outline multi-omic strategies for transcriptome
imputation from germline genetics for testing gene-trait associations
by prioritizing distal-SNPs to the gene of interest. In one extension,
we identify mediating biomarkers (CpG sites, microRNAs, and
transcription factors) highly associated with gene expression and
train predictive models for these mediators using their local
SNPs. Imputed values for mediators are then incorporated into the
final model as fixed effects with local SNPs to the
gene included as regularized effects. In the second extension, we
assess distal-eSNPs (SNPs in eQTLs) 
for their mediation effect through mediators local
to these distal-eSNPs. Highly mediated distal-eSNPs are then included
in the eventual transcriptomic prediction model. We show considerable gains in percent variance explained of gene
expression and TWAS power to detect gene-trait associations
using simulation analysis and real data
applications with TCGA breast cancer data and in ROS/MAP brain tissue
data.  This
integrative approach to transcriptome-wide imputation and association
studies aids in understanding the complex interactions underlying
genetic regulation within a tissue and identifying important risk
genes for various traits and disorders.
\end{abstract}

\begin{keyword}
TWAS\sep GWAS \sep eQTL \sep mediation analysis
\end{keyword}
