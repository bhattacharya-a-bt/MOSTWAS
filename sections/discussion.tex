Through a variety of simulations and 
real applications in two settings,
we demonstrate that multi-omic methods 
that prioritize distal variation in TWAS 
have higher predictive performance
and power to detect cell-specific gene-trait 
associations \cite{Brown2013IntegrativeEQTLs,Pierce2018Co-occurringMechanisms,vanderWijst2019Single-cellDisease},
especially when distal variation
contribute substantially to trait heritability.
We proposed two methods (MeTWAS
and DePMA) for
identifying and including distal
genetic variants in gene
expression prediction models.
We have provided implementations of these
methods in the MOSTWAS
(Multi-omic Strategies for Transcriptome-Wide
Association Studies)
R package, available freely
on Github.
MOSTWAS contains functions
to train expression models
with both MeTWAS and DePMA
and outputs models
with 5-fold cross-validation $R^2 \geq 0.01$
and significant germline heritability.
The package also contains functions and documentation
for simulation analyses
\cite{Mancuso2019ProbabilisticStudies},
the weighted burden \cite{Pasaniuc2014FastEnrichment,Gusev2016}
and follow-up permutation \cite{Gusev2016} and distal-SNPs
added-last tests for TWAS
using GWAS summary statistics,
and file-formatting. We also provide
guidelines for parallelization
to lessen computational time.

Not only does MOSTWAS improve
transcriptomic imputation both
in- and out-of-sample, it also provides a test for the 
identification of heritable mediators that affect
eventual transcription of the gene
of interest. These identified mediators
can provide insight into the
underlying mechanisms
for SNP-gene-trait associations
to improve
detection of gene-trait
associations and to prioritize biological units for functional
follow-up studies.
Using MOSTWAS and iCOGs
summary-level GWAS statistics
for breast cancer-specific survival \cite{Guo2015}, we
identified 11 survival-associated loci
that are enriched for p53 binding and
oxidoreductase activity 
pathways \cite{Bao2017P53Context,Zhou2016SystematicState}.
These loci include two genes 
(\textit{MAP3K6} and \textit{MAP4K5}) encoding
mitogen-activated protein kinases, which are 
signalling transduction molecules 
involved in 
the progression of aggressive
breast cancer hormone subtypes
\cite{Ahmad2016ClinicopathologicalCancers}.
TWAS using MOSTWAS models
was able to recapitulate 5 out of 14 known
Alzheimer's disease risk loci
in IGAP GWAS summary statistics
\cite{Lambert2013Meta-analysisDisease},
which were not recoverable
with local-only models.
We showed the utility of 
the distal-SNPs added-last test
to prioritize significant distal SNP-gene-trait
associations
for follow-up mechanistic studies,
which could not be identified 
using traditional local-only TWAS.
In PGC GWAS summary-level data
for major depressive disorder
\cite{Wray2018Genome-wideDepression},
we found 102 risk loci, 7 of which
were replicated in independent
GWAX summary statistics from the UK
Biobank \cite{Liu2017Case-controlDisease}.
Three of these seven loci (\textit{SYT1}, 
\textit{CACNA2D3}, 
\textit{ADAD2})
encode important proteins
involved in synaptic transmission in the brain
and RNA editing. Studies have shown that variation
at these loci may lead to loss
of function at synapses and RNA editing
that lead to psychiatric disorders
\cite{Ryan2006GeneGenes,Baker2015IdentificationCycling,Heyes2015GeneticDisorders,Savva2012TheFamily,Slotkin2013Adenosine-to-inosineDisease}.
All survival- or risk-associated loci identified by MOSTWAS
were not detected using local-only models.

A considerable limitation
of MOSTWAS is the 
increased computational burden
over local-only modelling, especially
in DePMA's permutation-based mediation
analysis for multiple genome-wide mediators.
We believe a Monte-Carlo resampling
method will aid
in scalability by making 
some standard
distributional assumptions
on the effect sizes of
SNPs and mediators in the DePMA
mediation model
\cite{Preacher2012AdvantagesEffects}.
Nevertheless, we believe that MOSTWAS's gain
in predictive performance 
and power to detect gene-trait associations
may outweigh this computational time.
In addition,
RNA-sequencing alignment errors
can lead to false positives in
distal-eQTL detection \cite{Saha2019FalseApproved}, 
and in turn,
bias the mediation modeling.
Cross-mapping estimation, as
described by Saha \etal{} \cite{Saha2019FalseApproved},
can be used to flag potentially falsely
positive distal-QTLs that are detected
as the first step in MeTWAS and DePMA.
Another limitation of MOSTWAS is
the general lack of rich multi-omic
panels, like TCGA-BRCA and ROS/MAP,
that provide a large set of
mediating biomarkers that
may be mechanistically involved
in gene regulation.
However, the two-step regression
framework outlined in MeTWAS
allows for importing
mediator intensity models
trained in other cohorts
to estimate
the germline portion
of total gene expression from
distal variants.

In conclusion,  
MOSTWAS provides a user-friendly
and intuitive tool that extends 
transcriptomic imputation and
association studies to include
distal genetic variants. 
We demonstrated that the methods
in MOSTWAS based on
two-step regression and mediation
analysis generally out-perform
local-only models in
both transcriptomic prediction and TWAS power,
though at the cost of longer computational times.
MOSTWAS enables users to utilize rich reference multi-omic 
datasets for enhanced gene 
mapping to better understand 
the genetic etiology of polygenic traits
and diseases with more direct insight into
functional follow-up studies.