

We first outline the two methods
proposed in this work, (1) mediator-enriched transcriptome-wide
prediction (MeTWAS) and (2) distal-eQTL 
prioritization via mediation analysis (DePMA). 
MeTWAS and DePMA are combined
in the MOSTWAS R package,
available freely at \url{https://bhattacharya-a-bt.github.io/MOSTWAS}.

\subsection{Mediator-enriched TWAS (MeTWAS)}

\subsubsection{Transcriptomic prediction using MeTWAS}

Here, we present mediator-enriched TWAS, or MeTWAS
one of the two tools presented in the MOSTWAS R package.
Across $n$ samples, consider the vector $Y_G$ 
of expression a gene $G$ 
of interest, the matrix $\mathbf{X}_G$ 
of local SNPs dosages
in a 500 kilobase window around gene $G$, and 
$m_G$ mediating biomarkers that
we estimate to be significantly associated
with the expression of gene $G$ via a
relevant one-way test of association. These
mediating biomarkers can be DNA methylation sites,
microRNAs, or transcription factors. Accordingly,
let the matrix $\mathbf{X}_{M_j}$ be the 
local-genotype dosages in a 500 kilobase
window around mediator $j,~1 \leq j \leq m_G$. Furthermore,
let $M_j$ be the intensity of mediator $j$ 
(i.e. methylation $M$-value if $j$ is a CpG site or
expression if $j$ is a miRNA or a gene).
Prior to any modelling, we scale $Y_G$ and all 
$M_j,~1 \leq j \leq m_G$ to zero mean and
unit variance. We also residualize
$M_j,~1 \leq j \leq m_G$ and $Y_G$ 
with the covariate matrix $\mathbf{X}_C$
to account for population stratification using
principal components of the global genotype matrix
and relevant clinical covariates to obtain
$\tilde{M}_j,~1 \leq j \leq m_G$ and $\tilde{Y}_G$.

Transcriptome prediction in MeTWAS draws from
two-step regression, as summarized in \textbf{Figure \ref{fig:ch4_fig1}}.
First, in the training set
for a given training-test split,
for $1 \leq j \leq m_G$, we model the residualized 
intensity $\tilde{M}_j$ of training-set specific 
mediator $j$ with
the following additive model:

\begin{equation}
	\tilde{M}_j = \mathbf{X}_{M_j,train}w_j +
	\varepsilon_m,
\end{equation}

\noindent where $w_j$ is the effect-sizes of
the SNPs in $\mathbf{X}_{M_j,train}$ on $\tilde{M}_j$
in the training set.
As in traditional transcriptomic
imputation models \cite{Gamazon2015,Gusev2016},
we find $\hat{w}_m$ using
one of the two following methods
with the largest predicted adjusted $R^2$: 
(1) elastic net regression with mixing parameter $\alpha = 0.5$
and $\lambda$ tuned over 5-fold cross validation
using glmnet \cite{Friedman2010}, or
(2) linear mixed modelling assuming random effects
for $\mathbf{X}_{M_j}$ using rrBLUP \cite{Endelman2011}.

For all $j$, using these optimized predictive
models for $M_j$ as denoted
by $\hat{w}_{M_j}$, 
we estimate the genetically regulated
intensity (GRIn) of the mediator $m_j$, denoted
GRIn$_{m_j}$, in the test set. Denote
$\mathbf{\hat{M}}$ as the $n \times m$
matrix of estimated GRIn, such that
the $j$th column of $\mathbf{\hat{M}}_j$ is
GRIn$_{m_j}$ across all $n$ samples. 

Next, we consider
the following additive model for
the residualized expression of gene $G$:

$$\tilde{Y}_G = \mathbf{\hat{M}}\beta_M + \mathbf{X}_Gw_G + \varepsilon_{Y_G},$$

\noindent where $\beta_M$ is the fixed effect-sizes
of GRIn$_{m_j}$ on $\tilde{Y}_G$, $\hat{\mathbf{M}}$
is the matrix of estimated GRIn for
all $m_j$ mediators, $\mathbf{X}_G$
are the local-SNPs to gene $G$, and $w_G$
are the ``random'' or regularized 
effect sizes of the local-SNPs.
We estimate $\beta_M$ by
traditional ordinary least squares, where
$\hat{\beta}_M = \left(\mathbf{\hat{M}}'\mathbf{\hat{M}}\right)^{-1}\mathbf{\hat{M}}'E_g$.
Next, using one of the methods outlined above
when estimating $\hat{w}_{M_j}$, we
can generate estimated effect sizes $\hat{w}_G$ 
of the local-SNPs on $\tilde{Y}_G$, residualized
with $\hat{\mathbf{M}}$.

\subsubsection{Transcriptomic imputation with MeTWAS}
In an external GWAS panel, if individual SNPs are 
available, we construct the mediator-enriched genetically regulated
expression (MeGReX) of gene $G$
directly using $\hat{w}_G$ and $(\hat{w}_j,\hat{\beta}_j),~1 \leq j \leq m_G$:

$$MeGReX_G = \sum_{j =two 1}^{m_G}{\mathbf{X}_{M_j,GWAS}\hat{w}_{M_j}\hat{\beta}_{M,j}} + 
\mathbf{X}_{G,GWAS}\hat{w}_G,$$

\noindent where $\mathbf{X}_{M_j,GWAS}$ and $\mathbf{X}_{G,GWAS}$ 
are the SNPs in the GWAS
panel local to mediator $j$ and gene $G$, respectively.
$MeGREX_G$ can be used in downstream tests of association. 


\subsection{Distal-eQTL prioritization via mediation analysis (DePMA)}

\subsubsection{Transcriptomic prediction using DePMA}

Expression prediction in DePMA hinges on
up-weighting distal-eQTLs to the gene of interest
via mediation analysis, adopting
methods from previous studies \cite{Pierce2014,Shan2019,Yang2017}. 
We first split data for
gene expression, SNP dosages, and any potential mediators
into $k$ training-testing splits. Based on the 
minor allele frequencies of SNPs and sample size,
we recommend a low number of splits (i.e. $k \leq 5$).

In the training set, we identify mediation test triplets
that consist of (1) a gene of interest $G$
with expression $Y_G$ (scaled to zero mean
and unit variance), (2)
a distal eSNP $s$ in association
with $G$ at a user-defined $P$-value
threshold (default of $P = 10^{-6}$)
with dosages $X_s$, and (3)
a set of $m$ biomarkers local 
to $s$ that are associated
with $s$ at a user-defined $P$-value threshold
(default of FDR-adjusted $P = 0.05$)
with intensities in the $m$ columns of 
$\mathbf{M}$. The columns of $\mathbf{M}$
are scaled to zero mean and unit variance.
Consider the following
mediation model for $1 \leq j \leq m$:

\begin{equation}\label{mediation}
\begin{split}
Y_G & = X_s\beta_s + \mathbf{M}\beta_{\mathbf{M}} + \mathbf{X}_C\beta_{C} + \varepsilon_{Y_G} \\
M_j & = X_s\alpha_{M_j} + \mathbf{X}_C\alpha_{C,j} + \varepsilon_{M_j}.
\end{split}
\end{equation}

Here, we have $\beta_{\mathbf{M}}$ as the
effects of the $M$ mediators local to $s$ on $Y_G$ adjusting
for the effects from $s$ and the covariates and 
$\alpha_{\mathbf{M}} = (\alpha_{M_1},\ldots,\alpha_{M_m})'$
as the effects of $s$ on mediators $M_j$, for $1 \leq j \leq m$.
We assume that $\varepsilon_{Y_G} \sim N(0,\sigma^2)$ and 
$\varepsilon_{\mathbf{M}} \sim \mathbf{N}_m\left(0,\mathbf{\Sigma}_M\right)$,
where $\mathbf{\Sigma}_M$ may have non-zero off-diagonal
elements that
represent non-zero covariance between mediator
intensities. 
Further, we assume that $\varepsilon_{Y_G}$ and
$\varepsilon_{\mathbf{M}}$ are independent.
We define the total mediation effect
(TME) 
\cite{Sobel1982AsymptoticModels} of SNP $s$
as 

$$TME = \alpha_{\mathbf{M}}^T\beta_{\mathbf{M}}.$$

\noindent We are interested in SNPs with large TME, which
we prioritize with the test of $H_0:~TME = 0$.
We assess this hypothesis with a permutation test, as
more direct methods of computing standard errors
for the estimated TME are often biased \cite{Mackinnon2004,Shan2019},
obtaining a permutation $P$-value.
We also provide an option to estimate
an asymptotic approximation to the
standard error of $TME$ 
and conduct a Wald-type test for
$TME = 0$. This asymptotic option 
is significantly faster at
the cost of inflated false positives 
(see \textbf{Supplemental Materials}
and 
\textbf{Supplemental Figure \ref{fig:ch4_fig2}}).
Corresponding to the $t$ testing triplets identified, 
we obtain a vectors of length $t$ of TMEs and $P$-values
for each distal eSNP to $G$. We estimate
Storey et al's $q$-value for each test to adjust
for multiple testing.
For the predictive model, we select distal
SNPs with $TME \neq 0$ 
at a given $q$-value threshold ($q < 0.10$
as a default)
and include them with all local SNPs in
a design matrix. We then find estimated SNP
weights using either elastic net or weighted least
squared regression.

\subsubsection{Transcriptomic imputation with DePMA}
In an external GWAS panel, if individual SNPs are 
available, we construct the genetically regulated
expression (GReX) of gene $G$
directly using $\hat{w}_G$ and $\hat{w}_t$:

$$GReX_G = \mathbf{X}_{t,GWAS}\hat{w}_{t} + 
\mathbf{X}_{G,GWAS}\hat{w}_G,$$

\noindent where $\mathbf{X}_{t,GWAS}$ is the matrix of dosages of the
$t$ distal-SNPs and $\mathbf{X}_{G,GWAS}$ is the matrix of 
dosages of the local SNPs to gene $G$ in the external GWAS
panel. $GREX_G$ can be used in downstream tests of association.
If individual SNPs are not available, the weighted
burden test can be employed using summary statistics with
permutation follow-up
test \cite{Gusev2016}.

\subsection{Tests of associations}


If individual SNPs are not available, then the weighted burden
$Z$-test proposed by Gusev et al can be employed \cite{Gusev2016} 
using
summary statistics. Briefly, we compute

\begin{equation}\label{wbtest}
\begin{split}
\tilde{Z} & = \frac{\mathbf{W}Z}{(\mathbf{W}\Sigma_{s,s}\mathbf{W}^T)^{1/2}}.
\end{split}
\end{equation}

Here, $Z$ is the vector of $Z$-scores of SNP-trait
associations for SNPs used in estimating $\hat{w}_{M_j}$
and $\hat{w}_G$. The matrix $W$ is defined as 
$\Sigma_{e,s}\Sigma_{s,s}^{-1}$, the product of
the covariance matrix between all SNPs
and the expression of gene $G$ and the covariance
matrix among all SNPs. These covariance matrices
are estimated from the reference panel used to estimate
$\hat{w}_{M_j}$
and $\hat{w}_G$.
and $\hat{w}_G$. The test statistic $\tilde{Z}$
can be compared to the standard Normal distribution
for inference.
We implement a permutation
test conditioning
on the GWAS effect sizes to assess
whether the same distribution
of $\hat{w}_G$ effect
sizes could yield a significant
association by chance \cite{Gusev2016}. 
We
permute $\hat{w}_G$ 1,000
times without replacement
and recompute the weighted
burden test statistic
to generate a permutation
null distribution for $\tilde{Z}$.
This permutation test
is only conducted for
overall associations
at a user-defined significance
level. 

Lastly, we also 
implement a test to assess
the information added from distal-eSNPs
in the weighted burden test beyond
what we find from local SNPs.
This added-last test
can be thought of as
a group added-last test
in regression analysis, applied
here to GWAS summary statistics.
Let $Z_l$ and 
$Z_d$  be the vector of $Z$-scores
from GWAS summary statistics
from
local and distal-SNPs identified
by a MOSTWAS model. 
The local
and distal-SNP effects
from the MOSTWAS model are represented
in $\mathbf{w}_l$ and 
$\mathbf{w}_d$.
Formally, we test
whether the weighted $Z$-score $\tilde{Z}_d \equiv \mathbf{w}_d^TZ_d$
from distal-SNPs is significantly
larger than 0 given the observed
weighted $Z$-score from local SNPs $\tilde{Z}_{l} \equiv \mathbf{w}_l^TZ_l$,
drawing from the assumption that
$(\tilde{Z}_l,\tilde{Z}_d)$
follow a bivariate Normal distribution. Namely, we
conduct a two-sided
Wald-type test for the null
hypothesis:

$$H_0: \mathbf{w}_d^T\mathbf{Z}_d | \mathbf{w}_l^T\mathbf{Z}_l = \tilde{Z}_{l,\text{obs}} = 0.$$

Under the null hypothesis,
we can derive that the
distribution of $\tilde{Z}_d|\tilde{Z}_l = \tilde{Z}_{l,\text{obs}}$
is normally distributed
with mean and variance
determined from the observed
local $\tilde{Z}_l$-score,
the SNP-effect size
vectors $\mathbf{w}_l$
and $\mathbf{w}_d$,
and components of the linkage
disequilibrium as estimated
from the reference panel \cite{Pasaniuc2014FastEnrichment}.
Full details and derivation
for this added-last test
are given in \textbf{Supplemental
Methods}.

\subsection{Simulation framework}

We first conducted simulations to assess
the predictive capability
and power to detect
gene-trait associations
under various phenotype ($h^2_p$),
local heritability of expression ($h^2_{e,l}$),
distal heritability of expression ($h^2_{e,d}$),
and proportion of causal local ($p_{c,l}$)
and distal ($p_{c,e}$) SNPs
for MeTWAS and DePMA.
We considered two scenarios for each
combination of $(h^2_p, h^2_{e,l}, h^2_{e,d}, p_{c,l},
p_{c,e})$: (1) the leveraged association
between the distal-SNP and gene of interest
exists in both the reference
and imputation panel, and (2) the leveraged association 
between distal-SNP and gene of interest
exists in the reference panel but is null
in the imputation panel.

Using TCGA data, we 
extracted 2,592 SNPs local to
the gene \emph{ESR1} 
on Chromosome 6 and 1,431 SNPs
local to the gene \emph{FOXA1}.
We generated (1) a reference panel with sample
size 400 with
simulated SNPs, expressions,
and one mediators 
and (2) a GWAS panel of
1,500 samples with simulated SNPs
and phenotypes using the following data
generating process, modified
from Mancuso et al's framework
\cite{Mancuso2019ProbabilisticStudies}:

We estimated the linkage disequilibrium $LD$
matrix of the SNPs $X_G$ with $n$
samples and $p$ SNPs, as follows
with regularization to ensure $LD$ is positive semi-definite:

$$LD = \frac{1}{n}X_G^TX_G + \frac{1}{10}I_p.$$ 

We computed the Cholesky decomposition of
$LD$ for faster sampling
\cite{Mancuso2019ProbabilisticStudies}.
We simulated SNPs for a 400-sample
reference panel $X_{g,ref}$ and 
1,500-sample GWAS panel $X_{g,GWAS}$.

We then simulated
effect sizes for $p_{c,l}$ of
the 2,592 local
SNPs $w_{g,l}$ 
from a standard Normal distribution.
We generated locally heritable
expression 

$$E_{g,l} = X_{G,ref}^Tw_{g,l} + \varepsilon_l,$$

with $\varepsilon_l \sim N(0,1-h^2_{e,l})$
and $w_{g,l}$ scaled to ensure the given $h^2_{e,l}$.
Similarly, we simulated effect sizes
for $p_{c,d}$ of the 1,431 distal-SNPs
$w_{g,d}$ and generated the distally heritable
intensity of the mediator $M_{g,d}$. We constructd
the distally heritable expression $E_{g,d}$
by scaling $M_{g,d}$ by $\beta \sim N(0,1)$ and adding random noise
that scales distal heritability to $h^2_{e,d}$.
We lastly formed the total expression $E_g = E_{g,l} + E{g,d}$.

Next, we simulated the phenotype
in the GWAS panel such that
the variance explained in the phenotype reflects
only that due to genetics. We drew a
causal effect size from gene expression $\alpha \sim N(0,1)$.
We computed the ``unobserved'' gene expression
in the GWAS panel as 

$$E_{g,GWAS} = X_{g,GWAS,local}^Tw_{g,l} + 
X_{g,GWAS,distal}^Tw_{g,d}\beta.$$

Here, we also considered a ``null'' case as well, where the
distal eQTLs are not detected in the GWAS panel (i.e. $w_{g,d} = 0$
for all distal-SNPs). GWAS summary statistics
were computed in this step for downstream weighted
burden testing.
We then fitted predictive models using MeTWAS, DePMA,
and local-only models (i.e. FUSION \cite{Gusev2016}),
computed the adjusted predictive $R^2$
in the reference panel, and tested
the gene-trait association in the GWAS panel
using a weighted burden test.
The association study power was defined
as the proportion of gene-trait association
tests with $P < 2.5 \times 10^{-6}$,
the Bonferroni-corrected significance threshold
for testing 20,000 independent genes.
With these simulated datasets,
we also assessed the power
of the distal added-last test
by computing the proportion
of significant distal associations
conditional on the local association
at FDR-adjusted $P < 0.05$.

\subsection{Data acquisition}

\subsubsection{Multi-omic data from TCGA-BRCA}

We retrieved genotype, RNA expression,
miRNA expression, and DNA methylation data
for breast cancer indications in The Cancer Genome Atlas.
Birdseed genotype files of 914 subject were downloaded
from the Genome Data Commons (GDC)
legacy (GRCh37/hg19) archive. Genotype
files were merged into a single binary PLINK
file format (BED/FAM/BIM) and imputed
using the October 2014 (v.3) release of the 1000 Genomes
Project dataset as a reference panel in
the standard two-stage imputation approach,
using SHAPEIT v2.87 for phasing and IMPUTE
v2.3.2 for imputation
\cite{OConnell2014,Delaneau2012,Howie2009}.
We excluded variants (1) with a minor
allele frequency of less that 1\% based on
genotype
dosage, (2) that deviated significantly from
Hardy-Weinberg equilibrium ($P < 10^{-8}$)
using appropriate functions in PLINK v1.90b3
\cite{Wigginton2005,Purcell2007}, and
(3) located on sex chromosomes.  
Final TCGA genotype data was coded as dosages,
with reference and alternative allele coding as in
dbSNP.

TCGA level-3 normalized RNA-seq expression data,
miRNA-seq expression data, and DNA methylation
data collected on Illumina Infinium HumanMethylation450 BeadChip were downloaded
from the Broad Institute's GDAC Firehose (2016/1/28
analysis archive). We intersected to the subset
of samples assayed for
genotype (4,564,962 variants), 
RNA-seq (15,568 genes), 
miRNA-seq (1,046 miRNAs),
and DNA methylation (485,578 CpG sites), 
resulting in a total
of 563 samples. We only consider
the autosome in our analyses.
We adjusted gene and miRNA expression
and DNA methylation by relevant
covariates (5 principal 
components of the genotype 
matrix, tumor stage at diagnosis,
and age).

\subsubsection{Multi-omic data from ROS/MAP}

We retrieved imputed genotype, RNA expression,
miRNA expression, and DNA methylation data
from The Religious Orders Study and Memory and Aging Project (ROS/MAP) Study
for samples derived from human pre-frontal cortex
\cite{A.Bennett2013OverviewStudy,A.Bennett2013OverviewProject,DeJager2012ADecline}.We excluded variants (1) with a minor
allele frequency of less that 1\% based on
genotype
dosage, (2) that deviated significantly from
Hardy-Weinberg equilibrium ($P < 10^{-8}$)
using appropriate functions in PLINK v1.90b3
\cite{Wigginton2005,Purcell2007}, and
(3) located on sex chromosomes. 
Final ROS/MAP genotype data was coded as dosages,
with reference and alternative allele coding as in
dbSNP. We intersectted to the subset
of samples assayed for 
genotype (4,141,537 variants), 
RNA-seq (15,857 genes), 
miRNA-seq (247 miRNAs),
and DNA methylation (391,626 CpG sites), resulting in a total
of 370 samples. We only consider
the autosome in our analyses.
We adjusted gene and miRNA expression
and DNA methylation by relevant
covariates (20 principal 
components of the genotype 
age at death,
and sex).

\subsubsection{Summary statistics
for downstream association studies}

For association testing, we downloaded
iCOGs GWAS summary statistics for
breast cancer-specific survival for
women of European ancestry \cite{Guo2015}.
Funding for BCAC and iCOGS came from: 
Cancer Research UK [grant numbers 
C1287/A16563, C1287/A10118, C1287/A10710, 
C12292/A11174, C1281/A12014, C5047/A8384, 
C5047/A15007, C5047/A10692, C8197/A16565], 
the European Union’s Horizon 2020 Research 
and Innovation Programme (grant numbers 
634935 and 633784 for BRIDGES and B-CAST respectively), the European Community’s 
Seventh Framework Programme under grant 
agreement n$^\circ$ 223175 
[HEALTHF2-2009-223175] (COGS), 
the National Institutes of Health 
[CA128978] and Post-Cancer GWAS 
initiative [1U19 CA148537, 1U19 
CA148065-01 (DRIVE) and 1U19 
CA148112 - the GAME-ON initiative], 
the Department of Defence [W81XWH-10-1-0341], 
and the Canadian Institutes of Health Research CIHR)
for the CIHR Team in Familial Risks of Breast 
Cancer [grant PSR-SIIRI-701]. All studies and 
funders as listed in Michailidou
et al \cite{Michailidou2013,Michailidou2015} 
and in Guo et al \cite{Guo2015} 
are acknowledged for their contributions.


For association testing, we downloaded
GWAS summary statistics for risk of
late-onset Alzheimer's disease 
from the International Genomics
of Alzheimer's Project (IGAP)
\cite{Lambert2013Meta-analysisDisease}.
We also downloaded GWAS and genome-wide
association by proxy (GWAX) summary statistics
for risk of major depressive
disorder (MDD) from the Psychiatric Genomics
Consortium \cite{Wray2018Genome-wideDepression}
and the UK Biobank \cite{Liu2017Case-controlDisease}, respectively.

IGAP is a large two-stage study based on GWAS 
on individuals of European ancestry. 
In stage 1, IGAP used genotyped and 
imputed data on 7,055,881 single 
nucleotide polymorphisms (SNPs) to 
meta-analyse four previously-published
GWAS datasets consisting of 17,008
Alzheimer's disease cases and 
37,154 controls (The European 
Alzheimer's disease Initiative –
EADI the Alzheimer Disease Genetics 
Consortium – ADGC The Cohorts for
Heart and Aging Research in Genomic 
Epidemiology consortium – CHARGE The 
Genetic and Environmental Risk in AD 
consortium – GERAD). In stage 2, 11,632
SNPs were genotyped and tested for 
association in an independent set of 
8,572 Alzheimer's disease cases and 
11,312 controls. Finally, a 
meta-analysis was performed combining results from stages 1 and 2.

\subsection{Model training
and association testing in
TCGA-BRCA and ROS/MAP}

Using both TCGA-BRCA and
ROS/MAP multiomic
data, we first identifed
associations between
SNPs and mediators (transcription
factor genes, miRNAs, and
CpG methylation sites),
mediators and gene expression,
and SNPs and gene expression
using MatrixEQTL
\cite{Shabalin2012a}.
These QTL analyses were
adjusted for 10 principal
components of the genotype
matrix to account for 
population stratification,
along with other relevant
covariates (tumor stage and age
for TCGA-BRCA, age and sex and smoking status for ROS/MAP).
For MeTWAS modeling, we considered
the top 5 mediators associated
with the gene of interest.
For DePMA models, we considered
all distal-SNPs
associated with gene expression
at raw $P < 10^{-6}$
and any local mediators at
FDR-adjusted $P < 0.05$.
Local windows for all models
were set to 0.5 Mb. For association
testing, we consider
only genes with significant 
non-zero
estimated total
heritability by
GCTA-LDMS \cite{Yang2015}
and cross-validation
adjusted $R^2 > 0$
across 5 folds.
The MeTWAS or DePMA
model with larger cross-validation
$R^2$ was considered
as the final MOSTWAS model
for a given gene. All
other modeling options
in MeTWAS and DePMA
were set to the defaults
provided by the MOSTWAS package.

Using TCGA-BRCA models,
we conducted TWAS burden testing
\cite{Pasaniuc2014FastEnrichment,Gusev2016} in iCOGs
GWAS summary statistics
for breast cancer-specific
survival in a cohort
of women of European ancestry.
We subjected TWAS-identified loci
at Benjamini-Hochberg \cite{Benjamini1995} 
FDR-adjusted $P < 0.05$
to permutation testing,
and any loci that persisted
past permutation testing to
distal variation
added-last testing.

Using ROS/MAP models,
we first conducted 
TWAS burden testing
in GWAS summary statistics
for late-onset
Alzheimer's disease risk
from IGAP, prioritized
14 known risk loci identified
from literature 
\cite{Lambert2013Meta-analysisDisease,Reitz2014GeneticDisease,Sims2017RareDisease,Yuan2017TheDisease}.
We subjected TWAS-identified loci
at Benjamini-Hochberg \cite{Benjamini1995} 
FDR-adjusted $P < 0.05$
to permutation testing,
and any loci that persisted
past permutation testing to
distal variation
added-last testing.
We similarly conducted TWAS
for risk of major
depressive disorder (MDD)
using GWAS summary statistics
from PGC (excluding data
from 23andMe and the UK Biobank)
with the necessary follow-up
tests. For any TWAS-identified loci
that persisted
permutation in PGC,
we further conducted TWAS
in GWAX summary statistics 
for MDD risk in
the UK Biobank
\cite{Liu2017Case-controlDisease}
for replication.
